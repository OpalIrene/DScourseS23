\documentclass{article}


% Set page size and margins
% Replace `letterpaper' with `a4paper' for UK/EU standard size
\usepackage[letterpaper,top=2cm,bottom=2cm,left=3cm,right=3cm,marginparwidth=1.75cm]{geometry}

% Useful packages
\usepackage{amsmath}
\usepackage{graphicx}
\usepackage[colorlinks=true, allcolors=blue]{hyperref}

\title{PS12}
\author{Opal Fraser}

\begin{document}
\maketitle


\section{6.}


%-------------
% table
%-------------

\input{table.tex}

At what rate are log wages missing? 
answer: 0.3068641 so, abotu 30 percent are missing
Do you think the logwage variable is most likely
to be MCAR, MAR, or MNAR?
answer: MAR, some are married and missing possibly due to being a new mom, having a child, off work. Some are union and missing, like if the wages for union jobs were above a certain threshold and not recorded, this could lead to missing logwage values for union jobs. Some are what seems to be in college years and missing. 



\section{7.}

The coefficient b1 represents the returns to schooling. 

The true value of ˆb1 = 0.091. Comment on the differences of ˆb1 across the models.
What patterns do you see? What can you conclude about the veracity of the various imputation methods?

a. complete cases only; b1=0.059042 rSq=0.03472 -statistically significant-error-0.009035; midrange rsq, midrange error
b. mean-logwage; b1=0.0362806 rSq=0.01808 -statistically significant-error-0.0062036; lower rSq, lower error
c. heckit-model; b1=0.091461 rSq=0.091461 -statistically significant-error-0.009789; higher rSq, higher error

Pattern is high rSq- higher error, lower rSq-smaller error. I would say the heckit model is the best of the three followed by complete cases only and last I would chose is mean-logwage. 
\input{table1.tex}



\end{document}