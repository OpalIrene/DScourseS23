\documentclass[12pt,english]{article}
\usepackage{mathptmx}

\usepackage[dvipsnames]{xcolor}
\definecolor{darkblue}{RGB}{0.,0.,139.}

\usepackage[top=1in, bottom=1in, left=1in, right=1in]{geometry}

\usepackage{amsmath}
\usepackage{amstext}
\usepackage{amssymb}
\usepackage{setspace}
\usepackage{lipsum}

\usepackage[authoryear]{natbib}
\usepackage{url}
\usepackage{booktabs}
\usepackage[flushleft]{threeparttable}
\usepackage{graphicx}
\usepackage[english]{babel}
\usepackage{pdflscape}
\usepackage[unicode=true,pdfusetitle,
 bookmarks=true,bookmarksnumbered=false,bookmarksopen=false,
 breaklinks=true,pdfborder={0 0 0},backref=false,
 colorlinks,citecolor=black,filecolor=black,
 linkcolor=black,urlcolor=black]
 {hyperref}
\usepackage[all]{hypcap} % Links point to top of image, builds on hyperref
\usepackage{breakurl}    % Allows urls to wrap, including hyperref

\linespread{2}

\begin{document}

\begin{singlespace}
\title{Further Examination of the Relationship Between Race and Gender in Mortgage Lending within the State of Oklahoma. \thanks{}}
\end{singlespace}

\author{Opal Fraser\thanks{Department of Economics, University of Oklahoma.\
E-mail~address:~\href{mailto:opalfraser@ou.edu}{opalfraser@ou.edu}}}

\date{\today}
\date{April 24, 2023}

\maketitle

\begin{abstract}
\begin{singlespace}
The project answers a question about lending in Oklahoma; do mortgage rates differ among race or gender? The difference in difference analysis looks at data before and after the COVID 19 pandemic. Using linear regression the data-set is analyzed to reveal differences. (notfinished). 

\end{singlespace}

\end{abstract}
\vfill{}

\pagebreak{}

\section{Introduction}\label{sec:intro}
The purpose of this report is to examine the relationship between the amount of a mortgage and the demographic characteristics of the borrower before and after the COVID 19 pandemic. The study uses data collected by the Federal Housing Finance Agency from a random sample of loan-level mortgage acquisitions acquired in 2019 and 2021. The analysis employs multiple linear regression models to estimate the impact of several independent variables, including race, gender, age, income, credit score, and loan-to-value ratio, on the dependent variable, the amount of the mortgage note/percentage rate.


\section{Literature Review}\label{sec:litreview}
Previous work by~\cite{gupta2022financial} shows that educational decisions are an important determinant of later-life earnings. This point is driven further in follow-up work by~\cite{gupta2022financial} and~\cite{gupta2022financial}.


\section{Data}\label{sec:data}
The primary data source for this research is the FHL Bank Public Use Database. Table 1 contains summary statistics.

\section{Empirical Methods}\label{sec:methods}
While my approach explores a number of different approaches, the primary empirical model can be depicted in the following equation:(I will code math properly for final report!)
[1]
LTV ~ lognoteamt + noteratepercent + LTV + bo1race*year + bo1gender*year + bo1age + debtexpenseratio 
[2]
lognoteamt ~ noteratepercent + LTV + bo1race*year + bo1gender*year + bo1age + debtexpenseratio + hsexpenseratio
[3]
noteratepercent ~ debtexpenseratio + lognoteamt*year + LTV + bo1race*year + bo1gender*year + bo1age
\begin{equation}
\label{eq:1}
Y_{it}=\alpha_{0} + \alpha_{1}Z_{it} + \alpha_{2} X_{it} + \varepsilon,
\end{equation}
where $Y_{it}$ is a continuous outcome variable for unit $i$ in year $t$, and $Z_{it}$ are characteristics about the firm at which $i$ is working, while $X_{it}$ are characteristics about $i$. The parameter of interest is $\alpha_{1}$.

\section{Research Findings}\label{sec:results}
The main results are reported Figure 1, 2, 3


\section{Conclusion}\label{secconclusion}
\begin{singlespace}
    Here I will add my conclusion on the regression analysis. I did not get this far since I've been doing the data search, wrangling and now the difference in difference analysis.
\end{singlespace}

\vfill
\pagebreak{}
\begin{spacing}{1.0}
\bibliographystyle{plain}
\bibliography{name.bib}
\addcontentsline{toc}{section}{References}
\end{spacing}

\vfill
\pagebreak{}
\clearpage

%========================================
% FIGURES AND TABLES 
%========================================
\section*{Figures and Tables}\label{sec:figTables}
\addcontentsline{toc}{section}{Figures and Tables}

\begin{figure}
\centering
\includegraphics[width=1\textwidth]{df1.jpeg}
\caption{\label{fig:df1}2019 data}
\end{figure}

\begin{figure}
\centering
\includegraphics[width=1\textwidth]{df2.jpeg}
\caption{\label{fig:df2}2021 data}
\end{figure}

\begin{figure}
\centering
\includegraphics[width=\textwidth]{regressionresults.jpeg}
\caption{\label{fig:results} regression results}
\end{figure}

\begin{figure}
\centering
\includegraphics[width=1\textwidth]{LTVres.jpg}
\caption{\label{fig:LTVres}LTV regression results}
\end{figure}

\begin{figure}
\centering
\includegraphics[width=1\textwidth]{lognoteamtres.jpg}
\caption{\label{fig:lognoteamount regression results}lognoteamount regression results}
\end{figure}

\begin{figure}
\centering
\includegraphics[width=\textwidth]{noteratepercentres.jpg}
\caption{\label{fig:note rate percentage regression results} note rate percentage regression results}
\end{figure}




\end{document}
